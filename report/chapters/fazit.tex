\chapter{Fazit}\label{chp:fazit} % Dani
\glsresetall

Die Motivation dieser Projektarbeit war es dynamische Wallpaper zu generieren, die sich je nach 
Tageszeit verändern. Dafür sollte jeweils ein Generative Adversarial Network (GAN) für das 
Erzeugen von Landschaftsbildern und das Manipulieren dieser Bilder 
implementiert werden. 

Leider gab es einige größere Hindernisse, wodurch gerade beim 
Manipulieren die Ergebnisse nicht praktisch einsetzbar sind. Zum einen ist es eine große 
Herausforderung einen geeigneten Bilderdatensatz zu finden bzw. zu erzeugen, zum 
anderen benötigt das Trainieren bei den hier verwendeten Netzen sehr viel Rechenleistung 
und moderne Hardware, um überhaupt zu laufen. Diese Hindernisse sind besonders beim 
Bildmanipulieren relevant, da dort die Netze größer sind und zwei Bilderdatensätze 
benötigt werden. Somit wurde aufgrund fehlender Bilder das Bild-manipulierende 
GAN mit Hunde- und Katzenbildern trainiert. Zudem war es mit der zur Verfügung 
stehenden Hardware nicht möglich komplexere Netze, wie das in \cref{sub:funit} 
beschriebene \gls{acr-funit} Netzwerk zu verwenden. Da die Trainingszeiten bei mehreren 
hundert Stunden liegen konnten, war es außerdem schwierig verschiedene Ansätze zu 
verfolgen und die Trainingsparameter zu optimieren. 

Das Erstellen eines großen Bilderdatensatzes für das Bildgenerieren war ebenfalls sehr 
aufwendig, da alle Landschaftsbilder-Datensätze viele schlechte Bilder enthielten und 
deshalb manuell gefiltert werden mussten. 

Für das Generieren von Landschaftsbildern wurde schlussendlich ein \gls{acr-SNDCGAN} 
verwendet. Dieses wurde mit über 7000 handsortierten Landschaftsbildern trainiert. 
Dabei sind die Ergebnisse durchaus als Landschaftsbilder zu erkennen, allerdings ebenfalls als künstlich generierte Bilder. Auffällig war, 
dass bei einer Bildauflösung von 128x72 Pixeln die Ergebnisse deutlich schlechter als 
mit der Auflösung 256x144 sind. Hier wäre es interessant das Training mit einer noch 
höheren Auflösung (auf besserer Hardware) durchzuführen und die Ergebnisse zu vergleichen. 
Zusätzlich zum  \gls{acr-SNDCGAN}  wurde auch ein \gls{acr-WGAN} implementiert. Da 
das \gls{acr-SNDCGAN} in ersten Versuchen jedoch besser abgeschnitten hat und die 
Hardwareressourcen begrenzt waren, wurde das \gls{acr-WGAN} nicht optimiert. Für das 
Projekt wäre es ebenfalls interessant gewesen, verschiedene GAN Architekturen 
zum Bildgenerieren zu vergleichen, dies hätte jedoch hier den Rahmen gesprengt.

Beim Manipulieren von Bildern sind die Ergebnisse deutlich schlechter als beim Generieren. 
Dies liegt vor allem daran, dass der verwendete Hunde- und Katzendatensatz nicht 
besonders groß ist und die Tiere von unterschiedlichen Richtungen in die Kamera schauen. 
Des weiteren konnte das CycleGAN, begingt durch den zu geringen Grafikspeicher, nur mit einer Batchgröße von vier 
trainiert werden. Dadurch fällt es dem Netz schwerer zu generalisieren. Anstatt eines 
CycleGANs wäre \gls{acr-funit} eventuell ebenfalls besser geeignet, dieses ist allerdings 
komplexer und schon das CycleGAN konnte kaum trainiert werden. Auch eine höhere 
Bildauflösung hätte helfen können, jedoch reichte der Grafikspeicher nur für eine 
Auflösung von 128x128 Pixeln.

Generative Adversarial Networks sind unglaublich mächtige neuronale Netzwerke, die eine Eingabe in eine 
antrainierte Datendomäne abbilden. Dabei sind sie allerdings sehr aufwendig zu trainieren.
Die Ergebnisse sind sehr stark von dem verwendeten Datensatz abhängig und um 
gute Resultate zu erzielen, wird sehr viel Rechenleistung benötigt.





