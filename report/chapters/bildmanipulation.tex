\chapter{Manipulieren von Bildern}\label{chp:bildmanipulation} %10 Seiten
\glsresetall

 \section{Modelle} % Dani
 
\subsection{CycleGAN}% Dani
 
 \subsection{FUNIT}% Dani
 
 \section{Implementierung} % Tim nach Implementierung Generieren
 
 \subsection{Umsetzung in Tensorflow/Keras}
 
 \subsection{Parameter Anpassung} % Haben wir noch nicht!!!!
 
 \section{Evaluierung} % TODO Tim und Joshua

 \paragraph{SIFID} Als Metrik für die Evaluierung der Ergebnisse der
 Bildmanipulation wird die \gls{acr-SIFID} verwendet, die von
 \citeauthor{shaham2019singan} in ihrem Paper \citetitle{shaham2019singan}
 \cite{shaham2019singan} eingeführt wurde. Dies ist eine Erweiterung der bereits
 für die Bildgenerierung verwendeten \gls{acr-FID} \cite{heusel2017gans} (vgl.
 \cref{evalGen} % TODO: Ist die da beschrieben?
 ). Anstatt die Differenz zwischen der Merkmalsverteilung für eine
 Gruppe von generierten und eine Gruppe von originalen Bildern zu messen (so wie
 die \gls{acr-FID}), misst die \gls{acr-SIFID} die Differenz zwischen einem
 Eingabebild in das I2I GAN und dem erzeugen Ausgabebild
 \cite[S. 4575]{shaham2019singan}. Wie die \gls{acr-FID} vergleicht die
 \gls{acr-SIFID} dabei die Aktivierung einer versteckten Schicht des
 \emph{Inception Networks} \cite{szegedy2015going}.
 %TODO: Ist das im Bilderzeugungs-Evaluierungskapitel so vorhanden?
 Anstatt allerdings die
 Aktivierung nach der letzten Pooling-Schicht zu verwenden, nutzt die
 \gls{acr-SIFID} die »tiefen« Eigenschaften, die in der Aktivierung der
 Convolution-Schicht vor der zweiten Pooling-Schicht zu finden sind \cite[S.
 4575]{shaham2019singan}. Wie für die \gls{acr-FID} sind niedrige Werte besser,
 sie bedeuten, dass der Unterschied zwischen den beiden Bildern geringer ist
 \cite[S. 5]{pang2021image}.