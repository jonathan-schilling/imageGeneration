\chapter{Datensatz}\label{chp:datensatz} %3 Seiten
\glsresetall
%Einleitung Tim
%Benötigt viele Bilder -> Woher?
Eine der größten Herausforderungen noch vor dem trainieren eines GANs ist das Beschaffen eines Datensatzes.  Dies liegt vor allem daran, dass die Qualität der Ergebnisse bei weniger Daten leidet. Laut  \citetitle{noauthor_nvidia_2020} \cite{noauthor_nvidia_2020} werden zwischen 50,000 und 100,000 Trainingsbilder benötigt um hoch qualitative Ergebnisse zu erzielen. Bei weniger Bildern kann es leichter dazu kommen, dass der Discriminator auswendig lernt. Für das Generieren wurden somit möglichst viele Landschaftsbilder gesucht. Bei Bildern für das manipulierende GAN, wurden zusätzlich verschiedene Klassen an Landschaftsbildern  benötigt, wie beispielsweise Tag/Nacht oder Sommer/Winter. Die Suche nach Datensätzen auf Kaggle, einer auf Data Science spezialisierte Plattform, auf der viele Datensätze zu finden sind, hat gezeigt, dass ein Datensatz in der hier benötigten Dimension schwer aufzutreiben ist.

\section{Bilderdatensatz mit Flickr} %Tim
Flickr ist ein online Fotodienst, bei dem Nutzer, die sich einen Account erstellt haben, ihre Bilder hochladen können um sie mit anderen zu teilen. Andere Benutzer haben nun die Möglichkeit diese zu kommentieren, zu bewerten oder weiterzuempfehlen \cite{noauthor_was_nodate}. Eine deutlich relevantere Funktion, damit Bilder von Flickr als Datensatz verwendet werden können, ist, dass Bildautoren die Fotos mit Tags versehen können. Dadurch bietet Flickr theoretisch zugriff auf Millionen von hand kategorisierte Bilder. 

Ein Manuelles herunterladen von mehreren tausend Landschaftsbildern ist allerdings viel zu aufwendig, weshalb das herunterladen nur mithilfe der Flickr API sinnvoll ist. Dabei setzt man jedoch großes Vertrauen auf die von anderen Nutzern gesetzten Kategorien (Tags). Nach dem Herunterladen von eintausend Bildern mit dem Tag \enquote{Landscape} wurde sichtbar, dass die Tags nicht zuverlässig genug sind. Es gab unscharfe und monochrome Bilder, Fotos auf denen Personen im Zentrum standen und Städteaufnahmen. Der Prozentsatz an verwertbaren Landschaftsbildern war viel zu niedrig. Eine Lösung für dieses Problem war es nicht nur nach Bildern mit dem Tag  \enquote{Landscape} zu suchen, sondern auch gewünschte Objekte durch eine Tag Blacklist auszusortieren. Somit werden Bilder mit Tags wie \enquote{City}, \enquote{Monochrome}, \enquote{Selfie} usw. noch vor dem Download aussortiert (Die genaue Liste an aussortierten Tags ist in der Datei tagsBlack.csv zu finden). Dies liefert deutlich bessere Ergebnisse, wobei dennoch einige Bilder nicht korrekt getaggt sind, wogen allerdings nicht viel unternommen werden kann. Zudem stellte sich heraus, dass die Durchlauffunktion der Flickr API manche Bilder mehrfach durchlief. Dieses Verhalten kombiniert mit dem Überspringen von Bildern, die mindestens einen Tag aus der Blacklist enthalten, sorgt dafür das der Download viel Zeit in Anspruch nimmt. Nach ungefähr zweieinhalbtausend Bildern wurden kaum noch neue Bilder gefunden (1 Bild pro 5 min), da die Rate an Bildern die sich wiederholen stetig anzusteigen scheint. Um einen besseren Datensatz zu erstellen, musste noch eine Landschaftsbildquelle gefunden werden.

\section{Vorhandene Bilderdatensätze}%Jonathan

\section{Datensatz Import und Verwendung}%Tim