\chapter{Einleitung}\label{chp:einleitung}%2 Seiten Jonathan
\glsresetall
%Hinführung
%Probelmstellung
%Ergebnisse 
%Aufbau

Seit einigen Jahren verfügen die macOS Versionen über dynamische Desktophintergründe. Diese Hintergrundbilder sind
Landschaften, welche zu verschiedenen Uhrzeiten abgebildet werden, und immer zur 
entsprechenden Tageszeit angezeigt werden. Dieses Feature lies die Frage aufkommen, wie 
solche Bildersets erstellt werden können, ohne jedoch zu jeder benötigten Uhrzeit ein Foto zu 
machen. 

Da sich neuronale Netze -- im konkreten Generative Adversarial Networks (GANs) -- gut für eine 
solche Anwendung eigenen, ist das Ziel dieser Projektarbeit, sich dem Thema zu nähern und eine 
Anwendung zu entwickeln, die ein Basisbild automatisch in eine andere Tageszeit transferiert. 

Das erste Ziel auf dem Weg zu einer solchen Anwendung war das grundsätzliche Generieren von 
Landschaftsbildern aus einem Zufallsvektor. Anhand dieser Aufgabe sollten erste Erfahrungen mit 
der Implementierung von GANs und der Beschaffung von Datensätzen gesammelt werden. Erst 
anschließend sollten kompliziertere Netzwerke implementiert werden, die dann das Manipulieren von 
Bildern ermöglichen.

Das Finden eines passenden Datensatzes war allerdings schwieriger als 
erwartet und auch die ersten Lernversuche zeigten, dass die zur Verfügung stehenden 
Rechenkapazitäten nicht für komplexe Netze reichen. Daher musste das Thema der Projektarbeit in 
\glqq Generieren und Manipulieren von Bildern mit GANs\grqq\ geändert werden und die Zielsetzung 
des Manipulierens der Tageszeit von Landschaftsbildern wurde zum Manipulieren von Hunde- bzw. 
Katzenbildern.

Unter diesem Thema wurden in der Projektarbeit verschiedene GAN-Architekturen untersucht und 
letztendlich ein SNDCGAN, sowie ein CycleGAN implementiert. Diese beiden Umsetzungen konnten 
erfolgreich trainiert werden, wobei die Ergebnisse des CycleGANs noch deutlich 
verbesserungsfähig sind. Das SNDCGAN hingegen liefert zufriedenstellende Bilder auf denen 
(schlecht aufgelöste) Landschaften zu erkennen sind.

Im Folgenden ist zunächst in \autoref{chp:forschungsstand} der Stand der Forschung beschrieben. 
Anschließend thematisiert das Kapitel \enquote{Datensatz} (\autoref{chp:datensatz}) alle 
erforderlichen Schritte, die nötig waren, um einen verwendbaren Datensatz herunterzuladen und 
vorzubereiten. Die \autoref{chp:bildgenerierung} und \ref{chp:bildmanipulation} behandeln 
jeweils die Modelle von neuronalen Netzen, wie auch deren Implementierung und Auswertung für 
das Generieren bzw. Manipulieren von Bildern. Abschließend werden die Ergebnisse dieser 
Projektarbeit im Fazit (\autoref{chp:fazit}) zusammengefasst.