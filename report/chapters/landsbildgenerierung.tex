 \chapter{Generierung von Landschaftsbildern}\label{chp:bildgenerierung} % 10 Seiten
 \glsresetall

 Der erste Teil des vorliegenden Projektes beschäftigt sich mit dem Generieren von neuen Landschaftsbildern aus Zufallsvektoren. Dazu wird das Konzept der in \cref{GANs} beschriebenen \gls{acr-GAN} verwendet.
 
 \section{Modelle}% Joshua

 Wie in \cref{GANs} beschreiben, gibt es im Bereich der \gls{acr-GAN}s zahlreiche
 Varianten und Anpassungen, die versuchen verschiedene Schwachstellen des
 Konzeptes auszugleichen. Um dieser Vielzahl von Möglichkeiten zu begegnen,
 wurden in diesem Projekt zwei weit verbreitete \cite[S. 3]{kurach2018gan} GAN-Architekturen umgesetzt, das \emph{\gls{acr-DCGAN}} \cite{radford2015unsupervised} zusammen mit dessen
 Erweiterung zum \emph{\gls{acr-SNDCGAN}} \cite{miyato2018spectral}, sowie das
 \emph{\gls{acr-WGAN}} \cite{arjovsky2017wasserstein}. Dies Erlaubt einen Vergleich der
 Vorgehensweisen und der Ergebnisse, siehe dazu \cref{evalGen}. % TODO: Ist das so?
 
\subsection{SNDCGAN} % Joshua

Das erste im Projekt umgesetzte \gls{acr-GAN} Modell, ist das \gls{acr-SNDCGAN}
genannte Modell, welches in dieser Form aus dem Übersichts-Paper
\citetitle{kurach2018gan} von \citeauthor{kurach2018gan} \cite{kurach2018gan}
übernommen ist. Dieses Modell ist eine Erweiterung des im Paper
\citetitle{radford2015unsupervised} von \citeauthor{radford2015unsupervised} \cite{radford2015unsupervised}
vorgestellten \gls{acr-DCGAN}s um die Änderungen des SN-GANs aus
\citetitle{miyato2018spectral} von \citeauthor{miyato2018spectral}
\cite{miyato2018spectral}.
Dies wird im Projekt durch die Standard-Kostenfunktion des ursprünglichen
\gls{acr-GAN}s \cite{goodfellow2014generative}, sowie einem leicht angepassten
Trainingsprozedere ergänzt.

\paragraph{\gls{acr-DCGAN}} Die Idee des \gls{acr-DCGAN} von
\citeauthor{radford2015unsupervised} \cite{radford2015unsupervised} ist es die
ursprünglich \cite[vgl.][]{goodfellow2014generative} auf vollständig verbundenen
Schichten (\emph{fully connected} bzw. \emph{Multilayer Perceptron (MLP)})
basierende \gls{acr-GAN} Architektur mit dem speziell für Bildverarbeitung
attraktiven \gls{acr-CNN} zu verbinden \cite[S. 1]{radford2015unsupervised}.
Dabei nutzt das \gls{acr-DCGAN} bestimmte (zu der Zeit) aktuelle Entwicklungen
der \gls{acr-CNN}s \cite[vgl.][S. 3]{radford2015unsupervised}: Auf deterministische Pooling-Funktionen wird zugunsten von
\emph{strided Convolutions} (d.\,h. Convolutions bei denen der Kernel um mehr als eins pro
Schritt bewegt wird) verzichtet; es werden bewusst gar keine (versteckten)
vollständig verbunden Schichten verwendet; es wird Batch-Normalisierung (d.\,h.
das Normalisieren der Eingabe zu einem Mittelwert von null und einer Varianz von
eins) auf die meisten Schichten angewandt und es wird \emph{ReLU} als
Aktivierungsfunktion (bis auf die Ausgabeschicht, die tanh verwendet) für den
Generator bzw. 
\emph{Leakey ReLU} für den Discriminator verwendet.
\gls{acr-DCGAN} beansprucht für sich, vor allem durch diese Änderungen, das
Training im Vergleich zu anderen Ansätzen stabilisieren zu können \cite[S.
9]{radford2015unsupervised}.

\paragraph{\gls{acr-SNDCGAN}} Das Paper \citetitle{miyato2018spectral}
\cite{miyato2018spectral} führt hauptsächlich eine neue
Gewichtsnormalisierungs-Technik ein, die aber für die Pro\-jekt-Imp\-le\-men\-tier\-ung
hier nicht relevant ist. Allerdings führt das Paper auch einige geringe
Änderungen an der Architektur des \gls{acr-DCGAN}s ein, die für die Architektur
des Projektes übernommen wurde. Dazu gehört hauptsächlich die Nutzung eines
acht-schichtigen Discriminators \cite[S. 3]{kurach2018gan}. Dies wird von
\citeauthor{kurach2018gan} \cite{kurach2018gan} als \gls{acr-SNDCGAN}
bezeichnet. Das \gls{acr-SNDCGAN} beansprucht für sich vor allem diversere
Datenpunkte zu erzeugen (d.\,h. den Zufall des Zufallsvektors besser umzusetzen)
und einen besseren \gls{acr-IS} (eine Metrik für das Beurteilen von
\gls{acr-GAN}s) zu erreichen \cite[S. 11]{miyato2018spectral}.

\paragraph{Architektur} Die Architektur des \gls{acr-SNDCGAN}s, so wie es in
diesem Projekt umgesetzt wird, stammt aus \citetitle{kurach2018gan}
\cite{kurach2018gan} und ist in \cref{tab:sndcgan} beschrieben. Dabei sieht man,
wie im Discriminator das Bild über Convolutions schrittweise verkleinert wird,
gleichzeitig aber mehr Kanäle, d.\,h. mehr Informationen über das Bild
hinzukommen. Der Generator andererseits verwendet
Deconvolutions\footnote{Korrekterweise handelt es sich hierbei eigentlich um
\emph{Transposed Convolutions} auch \emph{Fractionally Strided Convolutions}
genannt, nicht um Deconvolutions (d.\,h. die mathematische Umkehrung der
Convolution) im engeren Sinne; allerdings wird in der Literatur häufig dennoch
von Convolutions gesprochen, was hier übernommen wurde \cites[vgl.][S.
20]{dumoulin2016guide}[vgl.][S. 4]{radford2015unsupervised}.} um aus dem
eindimensionalen Zufallsvektor schrittweise ein Farbbild zu erzeugen.

\paragraph{Kostenfunktion} Als Kostenfunktion wird für die Projekt-Implementierung entsprechend des
\gls{acr-SNDCGAN} Papers die ursprüngliche Standard-Kostenfunktion aus dem Paper von
\citeauthor{goodfellow2014generative} \cite{goodfellow2014generative} verwendet. Diese ist
gegeben als $V(D, G) = \mathbb{E}_{x \sim p_{\text{data}}(x)} [\log D(x)] +
\mathbb{E}_{z \sim p_{z}(z)} [\log (1-D(G(z)))]$ \cite[S.
3]{goodfellow2014generative}. Dabei wird davon ausgegangen, dass der
Discriminator $D$ eine $1$ ausgibt, wenn er der Ansicht ist, dass das
eingegebene Bild sicher ein original Bild der Daten ist. Damit ist das Ziel des
Discriminators $V(D,G)$ zu maximieren, während der Generator die Funktion
minimieren möchte: $\min_{G} \max_{D} V(D,G)$. In der Praxis des Projektes wird,
ebenfalls dem \gls{acr-SNDCGAN} und \gls{acr-GAN} Papern folgend \cites[S.
3]{goodfellow2014generative}[S. 6]{miyato2018spectral}, statt der Minimierung
von $\log(1-D(G(z)))$, $\log D(G(z))$ maximiert, dies verringert das Problem von
verschwindenden Gradienten im frühen Training.

\paragraph{Training} Das Training der Projekt Implementierung folgt im
wesentlichen dem von \citeauthor{goodfellow2014generative}
\cite{goodfellow2014generative}. Der Generator und Discriminator werden jeweils
abwechselnd trainiert, wobei zuerst der Generator eine Batch von Beispiel-Daten
erzeugt, die anschließend vom Discriminator bewertet werden. Dies wird zuerst
für die Kosten des Generators verwendet und anschließend, zusammen mit einer
Bewertung von Originaldaten durch den Discriminator, für das Training des
Discriminators. Den Lernverfahren der einbezogenen Paper
\cite{radford2015unsupervised,miyato2018spectral,kurach2018gan} folgend,
verwendet die Implementierung des
Projektes den Adam-Optimizer \cite{kingma2014adam}.

\begin{table}[]
  \caption{SNDCGAN Architektur}
  \label{tab:sndcgan}
  \begin{center}
    \begin{minipage}{.5\linewidth}
      \caption{SNDCGAN Discriminator}
      \centering
        \begin{tabular}{lcl}
            \toprule
            Schicht & Kernel & Ausgabe\\
            \toprule
            Conv, lReLU & $[3,3,1]$ & $h \times w \times 64$\\
            \midrule
            Conv, lReLU & $[4,4,2]$ & $h/2 \times w/2 \times 128$\\
            \midrule
            Conv, lReLU & $[3,3,1]$ & $h/2 \times w/2 \times 128$\\
            \midrule
            Conv, lReLU & $[4,4,2]$ & $h/4 \times w/4 \times 256$\\
            \midrule
            Conv, lReLU & $[3,3,1]$ & $h/4 \times w/4 \times 256$\\
            \midrule
            Conv, lReLU & $[4,4,2]$ & $h/8 \times w/8 \times 512$\\
            \midrule
            Conv, lReLU & $[3,3,1]$ & $h/8 \times w/8 \times 512$\\
            \midrule
            Linear & - & $1$\\
            \bottomrule
        \end{tabular}
    \end{minipage}%
    \begin{minipage}{.5\linewidth}
      \centering
        \caption{SNDCGAN Generator}
        \begin{tabular}{lcl}
            \toprule
            Schicht & Kernel & Ausgabe\\
            \toprule
            $z$ & - & $128$\\
            \midrule
            Linear, BN, ReLU & - & $h/8 \times w/8 \times 512$\\
            \midrule
            Deconv, BN, ReLU & $[4,4,2]$ & $h/4 \times w/4 \times 256$\\
            \midrule
            Deconv, BN, ReLU & $[4,4,2]$ & $h/2 \times w/2 \times 128$\\
            \midrule
            Deconv, BN, ReLU & $[4,4,2]$ & $h \times w \times 64$\\
            \midrule
            Deconv, Tanh & $[3,3,1]$ & $h \times w \times 3$\\
            \bottomrule
        \end{tabular}
    \end{minipage} 
  \end{center}
  \begin{center}
    \bigskip
    \emph{Quelle:} Von \cite[S. 12]{kurach2018gan} übernommen.\\
    \emph{Legende:} Der Kernel ist beschrieben im Format
    $[\text{x\_Größe}, \text{y\_Größe}, \text{Schrittweite}]$;
    $h$ und $w$ beschreiben die Höhe bzw.
    Breite des Eingabe-Bildes in Pixeln; damit beschreibt die Ausgabe
    $\text{Höhe} \times \text{Breite} \times \text{Kanäle}$
    , wobei das Eingabe-Bild als
    farbiges Bild drei Kanäle hat.
  \end{center}
\end{table}
 
 \subsection{Wasserstein-GAN} % Tim
 
  \section{Implementierung} % Jonathan
 
 Der folgende Abschnitt thematisiert die Implementierung des zuvor beschriebenen SNDCGAN-Modells in Tensorflow bzw. Keras. Außerdem wird auch die Optimierung der Parameter der Netze behandelt.
 
 \subsection{Umsetzung in Tensorflow/Keras}\label{subsec:imp:sndc}
 
 Ein wichtiger Teil der Implementierung ist die Überführung der in Abschnitt~\ref{subsec:mod:sndc} vorgestellten Architektur des SNDCGANs in ein Tensorflow-Modell. Dafür wird das \glqq Sequential model\grqq\ von Keras~\cite{keras:SequentialModel} verwendet. In dem Modell werden die benötigten Schichten des Generators und des Discriminators definiert. Beim Trainieren wird der Input sequenziell von einer Schicht zur nächsten durchgereicht und adaptiert, daher auch die Namensgebung.  
 
 Nachdem die Architektur des SNDCGANs zu Beginn des Projektes implementiert war, konnten erste Tests durchgeführt werden. Dabei war es ein großes Problem, dass der Fehler des Discriminators schnell auf null ging und damit der Generator keine Chance mehr hatte, irgendetwas zu lernen bzw. zu verbessern, da jeder Versuch als künstliches Bild enttarnt wurde und der Generator somit keine Erfolge mehr verbuchen konnte. Aus diesem Grund wurde die Architektur des Discriminators angepasst und Dropout eingeführt. Dadurch werden bei jedem Trainingsvorgang nur ein Teil der Neuronen trainiert und der Generator bekommt eine Chance, sich gegen den Discriminator durchzusetzen.
 
 Die komplette Implementierung der am Ende verwendeten Architektur beider Modelle ist im Anhang unter \autoref{lst:sndcGenerator} und \ref{lst:sndcDiscriminator} zu finden.
 
 Weiter wird für das Trainieren des SNDCGANs eine Trainingsschleife benötigt. Die in diesem Fall verwendete Schleife wurde zunächst aus \cite{raschka2019} übernommen und im Folgenden an die Bedürfnisse des Projektes angepasst. Um die Gradienten für ein anschließendes Gradientenabstiegsverfahren zu berechnen, werden GradientTapes von Tensorflow~\cite{tf:gradientape} eingesetzt. Diese speichern alle Operationen, die bei der Ausführung des zu trainierenden Netzes durchgeführt werden und können daraus dann die Gradienten berechnen~\cite{tf:autodiff}.
 
 Um die neuronalen Netze zu optimieren, wird der Adam Algorithmus eingesetzt. Dieser führt einen Gradientenabstieg durch und wird bereits von Tensorflow zur Verfügung gestellt~\cite{tf:adam}. Als Besonderheit bringt Adam u. a. eine über die Zeit abnehmende Lernrate mit~\cite{kingma2014}.
 
 Ein weiterer wichtiger Teil der Implementierung ist die Möglichkeit, das Training zu pausieren und später wieder an gleicher Stelle aufzunehmen. Dies war vor allem wichtig, da die Rechenkapazitäten für dieses Projekt sehr begrenzt waren, somit das Trainieren viel Zeit in Anspruch genommen hat und das Training dabei nicht an einem Stück durchgeführt werden konnte. Deshalb wurde implementiert, dass in regelmäßigen Abständen Checkpoints des Lernfortschritts gespeichert werden. Diese Umfassen neben den Modellen des Generators und Discriminators auch deren Optimizer. Dies ist wichtig, da wie zuvor erwähnt wurde, die Lernrate der Adam Optimizer über die Zeit abnimmt. Somit ist nur eine Fortsetzung des Trainings beim gleichen Stand möglich, wenn auch die bisher verwendeten Optimizer geladen werden. 
 
 Zunächst war geplant, dass diese Checkpoints dauerhaft gespeichert werden, damit die trainierten Netzwerke später ausgewertet und verwendet werden können. Bei Testläufen hat sich allerdings herausgestellt, dass jeder Checkpoint fast 600 MB umfasst und dies sich dann über die ganzen Epochen zu einem großen Speicherverbrauch aufsummiert. Deshalb wurde zusätzlich eingeführt, dass die trainierten Modelle einzeln gespeichert werden und von den Checkpoints nur noch die letzten zwei erhalten bleiben. Dadurch konnte der Speicherplatz pro gespeicherter Epoche um zweidrittel reduziert werden.
 
 \subsection{Anpassung von Parametern} % Jonathan
 
 Eine große Herausforderung in diesem Projekt war die richtige Wahl der Parameter für das Training, sodass das neuronale Netz möglichst gut initialisiert wird, um dann gute Ergebnisse zu produzieren. Dabei war das Hauptproblem die fehlende Rechenkapazität, um verschiedene Konfigurationen ausführlich auszutesten, anschließend mit einer wissenschaftlichen Herangehensweise zu vergleichen und daraus die besten Parameter abzuleiten. Aus diesem Grund mussten die Entscheidungen auf Basis weniger Versuche getroffen werden. Daher ist es mit Sicherheit möglich, anhand einer ausführlicheren Analyse besser passende Parameter zu finden.
 
 Im Folgenden werden die vier Variablen \emph{Droprate}, \emph{initiale Lernrate}, \emph{Bildermenge} und \emph{Auflösung der Bilder} genauer thematisiert.
 
 Wie im vorherigen Abschnitt~\ref{subsec:imp:sndc} bereits erwähnt wurde, war es zu Beginn des Projektes ein großes Problem, dass der Fehler des Discriminators nach kurzer Zeit auf null ging und damit einen Lernfortschritt des Generators unmöglich machte. Neben der Einführung des Dropouts, war auch die Anpassung der initialen Lernrate auf der Seite des Discriminators wie auch auf der des Generators eine wichtige Stellschraube. Während es bei der Droprate auf Basis eines Blogbeitrags~\cite{brownlee2019} relativ schnell möglich war, einen Wert von 50 \% festzulegen, mussten bei der Lernrate einige Testläufe absolviert werden. 
 
 Die zu Beginn gewählten Lernraten bewegten sich in der Größenordnung von E-3 und mit ihnen trat das Verschwinden des Discriminator-Fehlers auf. Eine Vergrößerung der Lernrate wirkte sich deutlich negativ auf das Lernverhalten aus, da der Fehler des Discriminators noch schnell zu null ging. Als dritten Ansatz wurden unterschiedliche Lernraten für Generator und Discriminator getestet. Dabei wurde die des Generators im Bereich von E-3 deutlich größer gewählt als die des Discriminators mit E-4. Das Ziel war, den Generator schneller lernen zu lassen, als den Discriminator. Allerdings waren die Ergebnisse damit auch noch nicht zufriedenstellend. Erst ein Reduzieren beider Lernraten in den Bereich von E-4 hat für sichtbare Veränderungen bei den produzierten Beispielbildern gesorgt.
 
 Um den Trainingsprozess bei diesen Versuchen zu beschleunigen, wurde immer nur mit relativ wenig Bildern gelernt. Erst nachdem die ersten Erfolge durch eine bessere Wahl der initialen Lernrate sichtbar wurden, erhöhte sich die Bildermenge auf zunächst ca. 1.500 Stück und später dann auf etwas mehr als 7.000, was auch eine weitere Verbesserung der Ergebnisse mit sich brachte. Allerdings beschränkten sich die Verbesserung auf sichtbare Änderungen in den abgebildeten Formen, die zunehmend komplexer wurden. Die erstellten Bilder hatten jedoch noch wenig Ähnlichkeit zu (schlecht aufgelösten) Landschaftsbildern.
 
 Die letzte Änderung, die anschließend zu zufriedenstellenden Ergebnissen geführt hat, war eine Vergrößerung der Auflösung. Das bis dahin mit einer Bilderauflösung von 128x72 Pixeln trainierte SNDCGAN wurde auf 256x144 Pixel vergrößert. Dies bedeutete zwar einen deutlich Anstieg der Rechenzeit, allerdings war damit in den Beispielbildern mehr Inhalt zu erkennen, den das GAN lernen konnte.
 
 \section{Evaluierung}\label{evalGen} % TODO Jonathan, Dani