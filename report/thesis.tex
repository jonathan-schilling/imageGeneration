\NeedsTeXFormat{LaTeX2e}
\documentclass[a4paper,
10pt,
headsepline,           % Linie zw. Kopfzeile und Text
twoside,
openright,
pointlessnumbers,      % keine Punkte nach den letzten Ziffern in Überschriften
bibtotoc,              % LV im IV
chapterprefix,
DIV=9,                
%BCOR15mm               % Bindekorrektur
]{scrbook}
\KOMAoptions{DIV=last} % Neuberechnung Satzspiegel nach Laden von Paket helvet

\pagestyle{headings}
\usepackage{blindtext}

% für Texte in deutscher Sprache
\usepackage[ngerman]{babel}
\usepackage[utf8]{inputenc}
\usepackage[T1]{fontenc}

% Helvetica als Standard-Dokumentschrift
\usepackage[scaled]{helvet}
\renewcommand{\familydefault}{\sfdefault} 

% Literaturverzeichnis mit BibLKaTeX
\usepackage[babel,german=quotes]{csquotes}
\usepackage[backend=bibtex8,sorting=none,style=numeric-comp]{biblatex}
\DefineBibliographyStrings{ngerman}{
	andothers = {{et~al.}},
}
\bibliography{bibliography}
\renewcommand*{\bibfont}{\small}

\usepackage{tikz}
\tikzset{every picture/.style={line width=0.75pt}} 

\usepackage{fix-cm}
\usepackage{titlesec}

\usepackage{graphicx}
\usepackage{tabularx}
\usepackage{booktabs}
\usepackage{longtable}
\usepackage{multirow}
\usepackage{multicol}
\usepackage{float}
\usepackage{subfig}
\usepackage{dblfloatfix}

% Besondere Schriftauszeichnungen
\usepackage{xurl}              % \url{http://...} in Schreibmaschinenschrift
\usepackage{color}            % zum Setzen farbigen Textes
\usepackage{xcolor}
\usepackage{nicefrac}
\usepackage{amssymb, amsmath, amsthm} % Pakete für Mathe-Umgebungen und -Symbole

\definecolor{grayChapt}{gray}{0.65}
\titleformat{\chapter}[display]{\filleft\Huge\bfseries}{\vspace*{-5cm}\fontsize{100}{100}\selectfont\textcolor{grayChapt}\thechapter}{-1.9ex}{}[]%

\newtheorem{definition}{Definition}

\usepackage{setspace}         % Paket für div. Abstände, z.B. ZA
\setlength{\parindent}{0pt}   % kein linker Einzug der ersten Absatzzeile
\setlength{\parskip}{1.4ex plus 0.35ex minus 0.3ex} % Absatzabstand, leicht variabel

% Tiefe, bis zu der Überschriften in das Inhaltsverzeichnis kommen
\setcounter{tocdepth}{3}      % ist Standard

% Beispiele für Quellcode
\usepackage{listings}
\definecolor{codegreen}{rgb}{0,0.6,0}
\definecolor{codegray}{rgb}{0.5,0.5,0.5}
\definecolor{codepurple}{rgb}{0.58,0,0.82}
\definecolor{backcolour}{rgb}{0.95,0.95,0.92}
\lstset{
	backgroundcolor=\color{backcolour},   
	commentstyle=\color{codegreen},
	keywordstyle=\color{magenta},
	numberstyle=\tiny\color{codegray},
	stringstyle=\color{codepurple},
	basicstyle=\ttfamily\footnotesize,
	breakatwhitespace=false,         
	breaklines=true,                 
	captionpos=b,                    
	keepspaces=true,                 
	numbers=left,                    
	numbersep=5pt,                  
	showspaces=false,                
	showstringspaces=false,
	showtabs=false,                  
	tabsize=2
}
\renewcommand{\lstlistingname}{Quellcode}

\usepackage{booktabs}

% hier Namen etc. einsetzen
\newcommand{\fullnameA}{Joshua Hirschbrunn}
\newcommand{\emailA}{joshua.hirschbrunn@uni-ulm.de}
\newcommand{\fullnameB}{Jonathan Schilling}
\newcommand{\emailB}{jonathan.schilling@uni-ulm.de}
\newcommand{\fullnameC}{Tim Schneider}
\newcommand{\emailC}{tim-3.schneider@uni-ulm.de}
\newcommand{\fullnameD}{Daniela Tust}
\newcommand{\emailD}{daniela.tust@uni-ulm.de}
\newcommand{\titel}{Generieren und Manipulieren von Bildern mit GANs}
\newcommand{\jahr}{2022}
\newcommand{\matnrA}{123456}%TODO Matrikelnr. Joshua
\newcommand{\matnrB}{980556}
\newcommand{\matnrC}{123456}%TODO Matrikelnr. Tim
\newcommand{\matnrD}{123456}%TODO Matrikelnr. Dani
\newcommand{\gutachterA}{Prof.\,Dr.\,Friedhelm Schwenker}

% hier die Fakultät auswählen
\newcommand{\fakultaet}{Ingenieurwissenschaften, Informatik und Psychologie}
% hier das Institut einsetzen
\newcommand{\institut}{Institut für Neuroinformatik}

% Informationen, die LaTeX in die PDF-Datei schreibt
\pdfinfo{
  /Author (\fullnameA, \fullnameB, \fullnameC, \fullnameD)
  /Title (\titel)
  /Producer     (pdfeTex 3.14159-1.30.6-2.2)
  /Keywords ()
}

\definecolor{dark blue}{HTML}{000066}

\definecolor{todoCite}{HTML}{ABD4B6}%Quellen raussuchen
\definecolor{todoRef}{HTML}{A7C5DB}%Auf glossar, acronyme, bilder etc referenzieren
\definecolor{todoNorm}{HTML}{E6C59E}%Sonstige Todos
\definecolor{todoCont}{HTML}{B483C9}%Inhalte

\usepackage[hyperfootnotes=false]{hyperref}
\usepackage[multiple]{footmisc}
\usepackage[all]{hypcap}
\hypersetup{
pdftitle=\titel,
pdfsubject={Projektarbeit},
pdfproducer={pdfeTex 3.14159-1.30.6-2.2},
colorlinks=false,
linkcolor=dark blue,
citecolor=dark blue,
pdfborder=0 0 0	% keine Box um die Links!
}

\usepackage[toc,nonumberlist,nopostdot,nomain,acronym]{glossaries}\makeglossaries
%\newacronym[longplural={}]{acr-label}{short}{long}

%\newacronym{acr-wlan}{WLAN}{Wireless Local Area Network}
\newacronym{acr-KNN}{KNN}{Künstliche Neuronale Netze}
\newacronym{acr-GAN}{GAN}{Generative Adversarial Networks}
\newacronym[]{acr-I2I}{I2I}{Image-to-Image Translation}
\newacronym{acr-DCGAN}{DCGAN}{Deep Convolutional Generative Adverserial Network}
\newacronym{acr-SNDCGAN}{SNDCGAN}{Spectrally Normalized Deep Convolutional Generative Adverserial Network}
\newacronym[]{acr-WGAN}{WGAN}{Wasserstein Generative Adverserial Network}
\newacronym[]{acr-CNN}{CNN}{Convolutional Neural Network}
\newacronym[]{acr-IS}{IS}{Inception Score}
\newacronym[]{acr-SIFID}{SIFID}{Single Image Fréchet Inception Distance}
\newacronym[]{acr-FID}{FID}{Fréchet Inception Distance}

% Trennungsregeln
\hyphenation{Sil-ben-trenn-ung}

\begin{document}
\frontmatter

% Titelseite
\thispagestyle{empty}
\begin{addmargin*}[4mm]{-1mm}
	
	\includegraphics[height=1.8cm]{images/unilogo_bild}
	\hspace*{52.5mm}
	\includegraphics[height=1.8cm]{images/unilogo_wort}\\[1em]
	
	{\footnotesize
		{\bfseries Universität Ulm} \textbar ~89069 Ulm \textbar ~Germany
		\hspace*{57mm}\parbox[t]{36mm}{\bfseries Fakultät für 
			\fakultaet\\
			\mdseries \institut}\\[2cm]
		
		\parbox{143mm}{\bfseries \LARGE \titel}\\[2em]
		{\footnotesize Projektarbeit an der Universität Ulm}\\[2em]
		{\footnotesize \bfseries Vorgelegt von:}\\
		{\footnotesize \fullnameA\\ \emailA}\\[1em]
		{\footnotesize \fullnameB\\ \emailB}\\[1em]
		{\footnotesize \fullnameC\\ \emailC}\\[1em]
		{\footnotesize \fullnameD\\ \emailD}\\[2em]
		{\footnotesize \bfseries Gutachter:}\\                     
		{\footnotesize \gutachterA}\\\\
		{\footnotesize \jahr}
	}
\end{addmargin*}


% Impressum
\clearpage
\thispagestyle{empty}
{ \small
	\flushleft
	Fassung \today \\\vfill
	\copyright~\jahr~\fullnameA, \fullnameB, \fullnameC, \fullnameD\\[0.5em]
	Satz: PDF-\LaTeXe
}
% ab hier Zeilenabstand etwas größer 
\setstretch{1.2}

\tableofcontents

\mainmatter
\chapter{Einleitung}\label{chpt:einleitung}
\glsresetall


\appendix

\backmatter
\printglossaries

\clearpage
\thispagestyle{empty}

Name: \fullnameA \hfill Matrikelnummer: \matnrA \vspace{2cm}

\minisec{Erklärung}

Ich erkläre, dass ich die Arbeit selbständig verfasst und keine anderen als die angegebenen Quellen und Hilfsmittel verwendet habe.\vspace{2cm}

Ulm, den \dotfill

\hspace{10cm} {\footnotesize \fullnameA}

\clearpage
\thispagestyle{empty}

Name: \fullnameB \hfill Matrikelnummer: \matnrB \vspace{2cm}

\minisec{Erklärung}

Ich erkläre, dass ich die Arbeit selbständig verfasst und keine anderen als die angegebenen Quellen und Hilfsmittel verwendet habe.\vspace{2cm}

Ulm, den \dotfill

\hspace{10cm} {\footnotesize \fullnameB}

\clearpage
\thispagestyle{empty}

Name: \fullnameC \hfill Matrikelnummer: \matnrC \vspace{2cm}

\minisec{Erklärung}

Ich erkläre, dass ich die Arbeit selbständig verfasst und keine anderen als die angegebenen Quellen und Hilfsmittel verwendet habe.\vspace{2cm}

Ulm, den \dotfill

\hspace{10cm} {\footnotesize \fullnameC}

\clearpage
\thispagestyle{empty}

Name: \fullnameD \hfill Matrikelnummer: \matnrD \vspace{2cm}

\minisec{Erklärung}

Ich erkläre, dass ich die Arbeit selbständig verfasst und keine anderen als die angegebenen Quellen und Hilfsmittel verwendet habe.\vspace{2cm}

Ulm, den \dotfill

\hspace{10cm} {\footnotesize \fullnameD}
\end{document}
